
\documentclass{article}\usepackage[]{graphicx}\usepackage[]{color}
%% maxwidth is the original width if it is less than linewidth
%% otherwise use linewidth (to make sure the graphics do not exceed the margin)
\makeatletter
\def\maxwidth{ %
  \ifdim\Gin@nat@width>\linewidth
    \linewidth
  \else
    \Gin@nat@width
  \fi
}
\makeatother

\definecolor{fgcolor}{rgb}{0.345, 0.345, 0.345}
\newcommand{\hlnum}[1]{\textcolor[rgb]{0.686,0.059,0.569}{#1}}%
\newcommand{\hlstr}[1]{\textcolor[rgb]{0.192,0.494,0.8}{#1}}%
\newcommand{\hlcom}[1]{\textcolor[rgb]{0.678,0.584,0.686}{\textit{#1}}}%
\newcommand{\hlopt}[1]{\textcolor[rgb]{0,0,0}{#1}}%
\newcommand{\hlstd}[1]{\textcolor[rgb]{0.345,0.345,0.345}{#1}}%
\newcommand{\hlkwa}[1]{\textcolor[rgb]{0.161,0.373,0.58}{\textbf{#1}}}%
\newcommand{\hlkwb}[1]{\textcolor[rgb]{0.69,0.353,0.396}{#1}}%
\newcommand{\hlkwc}[1]{\textcolor[rgb]{0.333,0.667,0.333}{#1}}%
\newcommand{\hlkwd}[1]{\textcolor[rgb]{0.737,0.353,0.396}{\textbf{#1}}}%
\let\hlipl\hlkwb

\usepackage{framed}
\makeatletter
\newenvironment{kframe}{%
 \def\at@end@of@kframe{}%
 \ifinner\ifhmode%
  \def\at@end@of@kframe{\end{minipage}}%
  \begin{minipage}{\columnwidth}%
 \fi\fi%
 \def\FrameCommand##1{\hskip\@totalleftmargin \hskip-\fboxsep
 \colorbox{shadecolor}{##1}\hskip-\fboxsep
     % There is no \\@totalrightmargin, so:
     \hskip-\linewidth \hskip-\@totalleftmargin \hskip\columnwidth}%
 \MakeFramed {\advance\hsize-\width
   \@totalleftmargin\z@ \linewidth\hsize
   \@setminipage}}%
 {\par\unskip\endMakeFramed%
 \at@end@of@kframe}
\makeatother

\definecolor{shadecolor}{rgb}{.97, .97, .97}
\definecolor{messagecolor}{rgb}{0, 0, 0}
\definecolor{warningcolor}{rgb}{1, 0, 1}
\definecolor{errorcolor}{rgb}{1, 0, 0}
\newenvironment{knitrout}{}{} % an empty environment to be redefined in TeX

\usepackage{alltt}

\usepackage[utf8]{inputenc}
\usepackage{url}
\IfFileExists{upquote.sty}{\usepackage{upquote}}{}
\begin{document}

\thispagestyle{empty}
\begin{center}

\vspace{3cm}

\textsc{\Large École d'actuariat}\\
\textsc{\Large Université Laval}\\[0.5cm]

\vspace{5cm}

{ \LARGE \bfseries Travail pratique 1  \\ }

\vfill

\Large Olivier \textsc{Turcotte}\\
{\Large \textsc{Automne} 2018}

\end{center}
\newpage
\tableofcontents

\newpage

\section{Introduction}

Dans la section~\ref{sec:commandes}, on présente quelques commandes utiles, puis, des exemples de graphiques et tableaux sont donnés dans les sections \ref{sec:graph} et \ref{sec:tab}, respectivement.

\section{Quelques commandes utiles} 
\label{sec:commandes}

Les numéros de section et de sous-section se copieront directement dans la table des matières si on utilise la bonne commande. On peut référer aux sous-sections en leur assignant une étiquette, comme à la sous-section~\ref{ssec:listes}

\subsection{Listes et énumérations}
\label{ssec:listes}

On peut faire une énumération avec l'environnement suivant:

\begin{enumerate}
\item ceci est le premier point,

\item deuxième point, etc.
\end{enumerate}

\bigskip\noindent
Voici une autre liste:

\begin{itemize}
\item [a.] ceci est le premier point,

\item [b.] deuxième point, etc.
\end{itemize}

\subsection{Code}

On peut écrire du code avec l'environnement suivant:
\begin{knitrout}
\definecolor{shadecolor}{rgb}{0.969, 0.969, 0.969}\color{fgcolor}\begin{kframe}
\begin{alltt}
\hlstd{x} \hlkwb{<-} \hlkwd{rnorm}\hlstd{(}\hlnum{1}\hlstd{)}
\hlstd{x}
\end{alltt}
\begin{verbatim}
## [1] 1.508685
\end{verbatim}
\end{kframe}
\end{knitrout}

\subsection{Équations}

Dans le texte, on écrit les symboles mathématiques comme $\theta$. Si on veut mettre une équation en valeur, par exemple pour $j\in \{1,\ldots, n\}$, on a

$$
Y_j= \beta_0 +\beta_1 x_j +\varepsilon_j.
$$

On peut aussi écrire
\begin{equation}\label{eq1}
s^2 = \frac{1}{n-2} \sum_{i=1}^n \hat{\varepsilon}_i^2
\end{equation}
pour pouvoir référer à l'équation \ref{eq1} dans le texte. On ne numérote que les équations qui sont utilisées.



\section{Graphiques}
\label{sec:graph}


\section{Tableaux}
\label{sec:tab}


\appendix
\section{Annexe}

Voici une annexe.



\end{document}
